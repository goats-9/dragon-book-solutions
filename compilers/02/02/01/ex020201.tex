\begin{exercise}\label{ex:020201}
    Consider the context-free grammar
    \begin{align*}
        {S} \yld S\ S\ \texttt{+}\ |\ S\ S\ \texttt{*}\ |\ \texttt{a}
    \end{align*}
    \begin{enumerate}[label=\alph*)]
        \item Show how the string \texttt{aa+a*} can be generated by this grammar.
        \item Construct a parse tree for this string.
        \item What language does this grammar generate? Justify your answer.
    \end{enumerate}
\end{exercise}
\begin{solution}\label{sol:020201}
    \begin{enumerate}[label=\alph*)]
        \item We derive \texttt{aa+a*} using the following leftmost derivation.
        \begin{align*}
            S \yld SS\texttt{*} 
            \yld SS\texttt{+}S\texttt{*}
            \yld \texttt{a}S\texttt{+}S\texttt{*}
            \yld \texttt{aa+}S\texttt{*}
            \yld \texttt{aa+a*}
        \end{align*}
        \item The parse tree for the above string is in \autoref{fig:020201b}.
        \begin{figure}[!ht]
            \centering
            \begin{tikzpicture}
                [level distance = 2em, level/.style={sibling distance=5em/#1}]
                \node {\textit{S}}
                child {
                    node {\textit{S}}
                    child {
                        node {\textit{S}}
                        child {node {\texttt{a}}}
                    }
                    child {
                        node {\textit{S}}
                        child {node{\texttt{a}}}
                    }
                    child {node{\texttt{+}}}
                }
                child {
                    node {\textit{S}}
                    child {node{\texttt{a}}}
                }
                child {node{\texttt{*}}};
            \end{tikzpicture}
            \caption{Parse tree for the string \texttt{aa+a*}.}
            \label{fig:020201b}
        \end{figure}
        \item We claim that the given context-free grammar generates all and only 
        postfix expressions with operand \texttt{a} and operations \texttt{+} and 
        \texttt{*}. We prove this claim by induction on the number of operations
        in the expression, which we call $n$.
        
        \noindent Notice that at any step of the derivation, the length of the string 
        either increases by 2 or remains constant. Since the smallest string in this
        grammar is $S \yld \texttt{a}$, we see that any string generated by this 
        grammar has odd length.

        \noindent The base case ($n = 0$) consists of the derivation $S \yld 
        \texttt{a}$, so it holds.

        \noindent Suppose that the claim holds for all postfix expressions having 
        upto $n = k$ operations. Note that the last symbol for any postfix expression 
        must be an operation symbol by definition (except for the base case), 
        hence given any string $s = xy\texttt{b}$ with $n = k + 1$ operations, 
        we use the fact that $x$ and $y$ are postfix expressions with at most 
        $k$ operations and the induction hypothesis to arrive at the derivations
        \begin{align*}
            S \yld SS\texttt{b} 
            \drv xS\texttt{b} 
            \drv xy\texttt{b}
        \end{align*}
        where \texttt{b} is one of the two operators. This proves the claim.
    \end{enumerate}
\end{solution}