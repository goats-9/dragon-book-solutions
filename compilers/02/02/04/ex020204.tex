\begin{exercise}
    Construct unambiguous context-free grammars for each of the following 
    languages. In each case show that your grammar is correct.
    \begin{enumerate}[label=\alph*)]
        \item Arithmetic expressions in postfix notation.
        \item Left-associative lists of identifiers separated by commas.
        \item Right-associative lists of identifiers separated by commas.
        \item Arithmetic expressions of integers and identifiers with the four binary
        operators \texttt{+}, \texttt{-}, \texttt{*}, \texttt{/}.
        \item Add unary plus and minus to the operators of (d).
    \end{enumerate}
\end{exercise}
\begin{solution}
    \begin{enumerate}[label=\alph*)]
        \item The grammar is 
        \begin{align*}
            E &\yld E\ E\ op\ |\ num \\
            op &\yld \texttt{+}\ |\ \texttt{-}\ |\ \texttt{*}\ |\ \texttt{/} 
        \end{align*}
        We prove that each string generated by this grammar is generated 
        unambiguously by induction on the number of operations. For zero 
        operations, the expression $E \yld num$ is unambiguously generated.
        Suppose that all expressions with upto $k$ operations are 
        unambiguously generated. Then, any postfix operation with $k + 1$
        operations is of the form $E_1\ E_2\ op$ where $E_1$ and $E_2$ are
        unambiguously generated postfix expressions with upto $k$ operations.
        Thus, $E$ is derived unambiguously as
        \begin{align*}
            E \yld E\ E\ op \drv E_1\ E\ op \drv E_1\ E_2\ op
        \end{align*}
        which proves the claim.
        \item The grammar is $L \yld L,\ id\ |\ id$. We show that this grammar 
        is unambiguous by induction on the number of identifiers. For one 
        identifier, $L \yld id$ is derived unambiguously. Consider a list 
        $L = L_1,\ id$ of $k + 1$ identifiers. We derive $L$ as shown.
        \begin{align*}
            L \yld L,\ id \drv L_1,\ id
        \end{align*}
        which proves the claim.
        \item The grammar is $L \yld id,\ L\ |\ id$. Proof is similar to the 
        left associative case.
        \item The grammar is
        \begin{align*}
            E &\yld E\ \texttt{+}\ T\ |\ E\ \texttt{-}\ T\ |\ T \\
            T &\yld T\ \texttt{*}\ F\ |\ T\ \texttt{/}\ F\ |\ F \\
            F &\yld id\ |\ num\ |\ (E)
        \end{align*}
        We see that this grammar is unambiguous because of the fact that the 
        expression is built term by term from the rightmost end (represented 
        by $T$).
        \item The grammar is
        \begin{align*}
            E &\yld E\ \texttt{+}\ T\ |\ E\ \texttt{-}\ T\ |\ T \\
            T &\yld T\ \texttt{*}\ U\ |\ T\ \texttt{/}\ U\ |\ U \\
            U &\yld \texttt{+}\ F\ |\ \texttt{-}\ F\ |\ F \\
            F &\yld id\ |\ num\ |\ (E)
        \end{align*}
        Proof of unambiguity is similar to the above exercise.
    \end{enumerate}
\end{solution}