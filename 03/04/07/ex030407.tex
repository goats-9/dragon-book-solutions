\begin{exercise}\label{ex:030407}
    Show that the algorithm in \cref{code:030407} correctly tells whether the
    keyword is a substring of the given string. \emph{Hint}: proceed by
    induction on $i$. Show that for all $i$, the value of $s$ after line 4 is
    the length of the longest prefix of the keyword that is a suffix of
    $a_1a_2\ldots a_i$.

    \begin{listing}[!ht]
        \inputminted{c}{03/04/07/ex030407.c}
        \caption{The KMP Algorithm.}
        \label{code:030407}
    \end{listing}
\end{exercise}
\begin{solution}\label{sol:030407} 
    We claim that for all $i$, the value of $s$ after line 4 is the length of
    the longest prefix of the keyword that is a suffix of $a_1a_2\ldots a_i$. We
    prove this claim by induction on $i$.
    
    \noindent For $i = 1$, we must have $s = 1$ if $b_1 = a_1$ and $s = 0$
    otherwise, which proves the base case. Suppose the claim holds upto some
    $i$. Then, for $i + 1$, we need to find the maximal $s$ such that $a_{i-s+k}
    = b_k,\ 1 \le k \le s+1$. The \texttt{while} loop in line 3 reduces
    \texttt{s} to this maximal value. If no such value of $s$ is found, then we
    must have $s = 0$. In either case, the induction hypothesis holds, and the
    claim is proved.

    \noindent At each iteration of the \texttt{for} loop, we check whether $s =
    n$. This implies that keyword $b$ is a suffix of $a_1a_2\ldots a_i$ for some
    $i$. Therefore, we have found a match. If this situation does not occur,
    control exits the \texttt{for} loop and there is no match. This proves the
    correctness of the KMP algorithm.
\end{solution}