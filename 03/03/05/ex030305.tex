\begin{exercise}\label{ex:030305}
    Write regular definitions for the following languages:
    \begin{enumerate}[label=\alph*)]
        \item All strings of lowercase letters that contain the five vowels in 
        order.
        \item All strings of lowercase letters in which the letters are in 
        ascending lexicographic order.
        \item Comments, consisting of a string surrounded by \texttt{/*} and 
        \texttt{*/}, without an intervening \texttt{*/}, unless it is inside 
        double-quotes (\texttt{"}).
        \item All strings of digits with no repeated digits. \emph{Hint}: Try 
        this problem first with a few digits, such as $\cbrak{0,1,2}$.
        \item All strings of digits with at most one repeated digit.
        \item All strings of \texttt{a}'s and \texttt{b}'s with an even number 
        of \texttt{a}'s and an odd number of \texttt{b}'s. 
        \item The set of Chess moves, in the informal notation, such as
        \texttt{{p-k4}} or \texttt{kbp}$\times$\texttt{qn}.
        \item All strings of \texttt{a}'s and \texttt{b}'s that do not contain
        the substring \texttt{abb}.
        \item All strings of \texttt{a}'s and \texttt{b}'s that do not contain
        the subsequence \texttt{abb}.
    \end{enumerate}
\end{exercise}
\begin{solution}\label{sol:030305}
    \begin{enumerate}[label=\alph*)]
        \item The regular expression is \texttt{C*a[Ca]*e[Ce]*i[Ci]*o[Co]*u[Cu]*}.
        \texttt{C} is the shorthand for all consonants \texttt{[\^{}aeiou]}.
        \item The regular expression is \texttt{a*b*\dots z*}.
        \item The regular expression is \texttt{\textbackslash/\textbackslash*%
        (\textbackslash".*\textbackslash"|%
        [\^{}\textbackslash*\textbackslash"]*|%
        \textbackslash*[\^{}\textbackslash/]*)*%
        \textbackslash*\textbackslash/}.
        \item We describe the regular definition by describing an appropriate NFA
        $N\brak{Q, \Sigma, \delta, q_s, F}$.
        \begin{enumerate}[label=\roman*)]
            \item $Q = \cbrak{q_s, q_0, \ldots, q_9}$, where each state except $q_s$ 
            represents the last scanned digit.
            \item $\Sigma = \{\texttt{0-9}\}$.
            \item $\delta$ is given by
            \begin{align*}
                \delta\brak{q, i} = 
                \begin{cases}
                    \cbrak{q_i} & q = q_s,\ i \in \Sigma \\
                    \cbrak{q_i} & q = q_j,\ j \in \Sigma,\ i \neq j
                \end{cases}
            \end{align*}
            \item $q_0 = q_s$.
            \item $F = Q \setminus \cbrak{q_s}$ are the accepting states.
        \end{enumerate}
        \item Similar to the above expression, we describe the regular definition 
        by describing an appropriate NFA $N\brak{Q, \Sigma, \delta, q_s, F}$.
        \begin{enumerate}[label=\roman*)]
            \item $Q = \cbrak{q_s, q_0, q_0', \ldots, q_9, q_9'}$, where each state 
            except $q_s$ represents the last scanned digit. Additionally, the states
            marked with a \texttt{'} denote that there has been a repeated digit.
            \item $\Sigma = \{\texttt{0-9}\}$.
            \item $\delta$ is given by
            \begin{align*}
                \delta\brak{q, i} = 
                \begin{cases}
                    \cbrak{q_i} & q = q_s,\ i \in \Sigma \\
                    \cbrak{q_i'} & q = q_i,\ i \in \Sigma \\
                    \cbrak{q_i} & q = q_j,\ j \in \Sigma,\ i \neq j \\
                    \cbrak{q_i'} & q = q_j',\ j \in \Sigma,\ i \neq j
                \end{cases}
            \end{align*}
            \item $q_0 = q_s$.
            \item $F = Q \setminus \cbrak{q_s}$ are the accepting states.
        \end{enumerate}
        \item We describe a DFA $M(Q, \Sigma, \delta, q_0, F)$ that recognizes 
        the required language.
        \begin{enumerate}[label=\roman*)]
            \item $Q = \cbrak{q_{ij}\ |\ 0 \le i, j \le 1}$. Here, the first 
            index represents the parity of \texttt{a} and the second index 
            represents the parity of \texttt{b} in the string.
            \item $\Sigma = \cbrak{\texttt{a-b}}$.
            \item $\delta$ is given in \autoref{tab:030305f}.
            \begin{table}[!ht]
    \centering
    \begin{tabular}{|c|c|c|}
        \hline
        $\delta\brak{q, a}$ & \texttt{a} & \texttt{b} \\
        \hline
        $q_{00}$ & $q_{10}$ & $q_{01}$ \\
        \hline
        $q_{01}$ & $q_{11}$ & $q_{00}$ \\
        \hline
        $q_{10}$ & $q_{00}$ & $q_{11}$ \\
        \hline
        $q_{11}$ & $q_{01}$ & $q_{10}$ \\
        \hline
    \end{tabular}
    \caption{$\delta(q, a)$ for DFA $M$.}
    \label{tab:030305f}.
\end{table}
            \item $q_0 = q_{00}$.
            \item $F = \cbrak{q_{01}}$.
        \end{enumerate}
        \item 
        \item The regular expression is \texttt{b*(a+b?)*}.
        \item The regular expression is \texttt{b*a*b?a*}.
    \end{enumerate}
\end{solution}