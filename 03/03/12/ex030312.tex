\begin{exercise}\label{ex:030312}
    SQL allows a rudimentary form of patterns in which two characters have 
    special meaning: underscore (\texttt{\textunderscore}) stands for any one 
    character and percent-sign (\texttt{\%}) stands for any string of 0 or more 
    characters. In addition, the programmer may define any character, say $e$, 
    to be the escape character, so an $e$ preceding , \texttt{\textunderscore}, 
    \texttt{\%}, or another $e$ gives the character that follows its literal 
    meaning. Show how to express any SQL pattern as a regular expression, given 
    that we know which character is the escape character.
\end{exercise}
\begin{solution}\label{sol:030312}
    Using \texttt{Lex} regular expression, we have the following equivalent 
    regular expressions in \autoref{tab:030312}.
    \begin{table}[!ht]
    \centering
    \begin{tabular}{l|l}
        \hline
        \hline
        \textsc{SQL Pattern} & \textsc{Regular Expression} \\
        \hline
        \texttt{\textunderscore} & \texttt{.} \\
        \texttt{\%} & \texttt{.*} \\
        $ec$ & \textbackslash$c$ \\
        \hline
    \end{tabular}
    \caption{Equivalent regular expressions for SQL patterns.}
    \label{tab:030312}
\end{table}
\end{solution}