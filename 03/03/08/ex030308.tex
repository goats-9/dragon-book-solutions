\begin{exercise}\label{ex:030308}
    In \texttt{Lex}, a \emph{complemented character} class represents any 
    character except the ones listed in the character class. We denote a 
    complemented class by using \texttt{\^{}} as the first character; this 
    symbol (caret) is not itself part of the class being complemented, unless 
    it is listed within the class itself. Thus, \texttt{[\^{}A-Za-z]} matches 
    any character that is not an uppercase or lowercase letter, and 
    \texttt{[\^{}\textbackslash\^{}]} represents any character but the caret 
    (or newline, since newline cannot be in any character class). Show that for 
    every regular expression with complemented character classes, there is an 
    equivalent regular expression without complemented character classes.
\end{exercise}
\begin{solution}\label{sol:030308}
    We can simply substitute the complemented character classes with character
    classes that contain all but the complemented characters. Since the character
    class of a language is finite, we can always do this, and not change the 
    meaning of the regular expression.
\end{solution}