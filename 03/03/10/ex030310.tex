\begin{exercise}\label{ex:030310}
    The operator \texttt{\^{}} matches the left end of a line, and \texttt{\$} 
    matches the right end of a line. The operator \texttt{\^{}} is also used to 
    introduce complemented character classes, but the context always makes it 
    clear which meaning is intended. For example, \texttt{\^{}[\^{}aeiou]*\$} 
    matches any complete line that does not contain a lowercase vowel.
    \begin{enumerate}[label=\alph*)]
        \item How do you tell which meaning of \^{} is intended?
        \item Can you always replace a regular expression using the \^{} and \$ 
        operators by an equivalent expression that does not use either of these 
        operators?
    \end{enumerate}
\end{exercise}
\begin{solution}\label{sol:030310}
    \begin{enumerate}[label=\alph*)]
        \item We can distinguish the two meanings by checking whether the caret 
        is present inside a character class right after the opening square bracket.
        If so, then the caret denotes negation, otherwise it denotes the start of a 
        line.
        \item 
    \end{enumerate}
\end{solution}